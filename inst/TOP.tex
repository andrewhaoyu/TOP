\documentclass[11pt,]{article}
\usepackage[left=1in,top=1in,right=1in,bottom=1in]{geometry}
\newcommand*{\authorfont}{\fontfamily{phv}\selectfont}
\usepackage[]{mathpazo}


  \usepackage[T1]{fontenc}
  \usepackage[utf8]{inputenc}



\usepackage{abstract}
\renewcommand{\abstractname}{}    % clear the title
\renewcommand{\absnamepos}{empty} % originally center

\renewenvironment{abstract}
 {{%
    \setlength{\leftmargin}{0mm}
    \setlength{\rightmargin}{\leftmargin}%
  }%
  \relax}
 {\endlist}

\makeatletter
\def\@maketitle{%
  \newpage
%  \null
%  \vskip 2em%
%  \begin{center}%
  \let \footnote \thanks
    {\fontsize{18}{20}\selectfont\raggedright  \setlength{\parindent}{0pt} \@title \par}%
}
%\fi
\makeatother




\setcounter{secnumdepth}{0}

\usepackage{color}
\usepackage{fancyvrb}
\newcommand{\VerbBar}{|}
\newcommand{\VERB}{\Verb[commandchars=\\\{\}]}
\DefineVerbatimEnvironment{Highlighting}{Verbatim}{commandchars=\\\{\}}
% Add ',fontsize=\small' for more characters per line
\usepackage{framed}
\definecolor{shadecolor}{RGB}{248,248,248}
\newenvironment{Shaded}{\begin{snugshade}}{\end{snugshade}}
\newcommand{\KeywordTok}[1]{\textcolor[rgb]{0.13,0.29,0.53}{\textbf{#1}}}
\newcommand{\DataTypeTok}[1]{\textcolor[rgb]{0.13,0.29,0.53}{#1}}
\newcommand{\DecValTok}[1]{\textcolor[rgb]{0.00,0.00,0.81}{#1}}
\newcommand{\BaseNTok}[1]{\textcolor[rgb]{0.00,0.00,0.81}{#1}}
\newcommand{\FloatTok}[1]{\textcolor[rgb]{0.00,0.00,0.81}{#1}}
\newcommand{\ConstantTok}[1]{\textcolor[rgb]{0.00,0.00,0.00}{#1}}
\newcommand{\CharTok}[1]{\textcolor[rgb]{0.31,0.60,0.02}{#1}}
\newcommand{\SpecialCharTok}[1]{\textcolor[rgb]{0.00,0.00,0.00}{#1}}
\newcommand{\StringTok}[1]{\textcolor[rgb]{0.31,0.60,0.02}{#1}}
\newcommand{\VerbatimStringTok}[1]{\textcolor[rgb]{0.31,0.60,0.02}{#1}}
\newcommand{\SpecialStringTok}[1]{\textcolor[rgb]{0.31,0.60,0.02}{#1}}
\newcommand{\ImportTok}[1]{#1}
\newcommand{\CommentTok}[1]{\textcolor[rgb]{0.56,0.35,0.01}{\textit{#1}}}
\newcommand{\DocumentationTok}[1]{\textcolor[rgb]{0.56,0.35,0.01}{\textbf{\textit{#1}}}}
\newcommand{\AnnotationTok}[1]{\textcolor[rgb]{0.56,0.35,0.01}{\textbf{\textit{#1}}}}
\newcommand{\CommentVarTok}[1]{\textcolor[rgb]{0.56,0.35,0.01}{\textbf{\textit{#1}}}}
\newcommand{\OtherTok}[1]{\textcolor[rgb]{0.56,0.35,0.01}{#1}}
\newcommand{\FunctionTok}[1]{\textcolor[rgb]{0.00,0.00,0.00}{#1}}
\newcommand{\VariableTok}[1]{\textcolor[rgb]{0.00,0.00,0.00}{#1}}
\newcommand{\ControlFlowTok}[1]{\textcolor[rgb]{0.13,0.29,0.53}{\textbf{#1}}}
\newcommand{\OperatorTok}[1]{\textcolor[rgb]{0.81,0.36,0.00}{\textbf{#1}}}
\newcommand{\BuiltInTok}[1]{#1}
\newcommand{\ExtensionTok}[1]{#1}
\newcommand{\PreprocessorTok}[1]{\textcolor[rgb]{0.56,0.35,0.01}{\textit{#1}}}
\newcommand{\AttributeTok}[1]{\textcolor[rgb]{0.77,0.63,0.00}{#1}}
\newcommand{\RegionMarkerTok}[1]{#1}
\newcommand{\InformationTok}[1]{\textcolor[rgb]{0.56,0.35,0.01}{\textbf{\textit{#1}}}}
\newcommand{\WarningTok}[1]{\textcolor[rgb]{0.56,0.35,0.01}{\textbf{\textit{#1}}}}
\newcommand{\AlertTok}[1]{\textcolor[rgb]{0.94,0.16,0.16}{#1}}
\newcommand{\ErrorTok}[1]{\textcolor[rgb]{0.64,0.00,0.00}{\textbf{#1}}}
\newcommand{\NormalTok}[1]{#1}


\title{Two-stage polytmous logistic regression tutorial  }



\author{\Large Haoyu Zhang\vspace{0.05in} \newline\normalsize\emph{Department of Biostatistics, Johns Hopkins University, Baltimore, MD,
U.S.A.}   \and \Large Ni Zhao\vspace{0.05in} \newline\normalsize\emph{Department of Biostatistics, Johns Hopkins University, Baltimore, MD,
U.S.A.}   \and \Large Thomas U. Ahearn\vspace{0.05in} \newline\normalsize\emph{National Cancer Institute, Division of Cancer Epidemiology and Genetics,
Rockville, MD, U.S.A.}   \and \Large William Wheeler\vspace{0.05in} \newline\normalsize\emph{Information Management Services, Inc., Rockville, MD, U.S.A}   \and \Large Montserrat García-Closas\vspace{0.05in} \newline\normalsize\emph{National Cancer Institute, Division of Cancer Epidemiology and Genetics,
Rockville, MD, U.S.A.}   \and \Large Nilanjan Chatterjee\vspace{0.05in} \newline\normalsize\emph{Department of Biostatistics, Johns Hopkins University, Baltimore, MD,
USA}  }


\date{}

\usepackage{titlesec}

\titleformat*{\section}{\normalsize\bfseries}
\titleformat*{\subsection}{\normalsize\itshape}
\titleformat*{\subsubsection}{\normalsize\itshape}
\titleformat*{\paragraph}{\normalsize\itshape}
\titleformat*{\subparagraph}{\normalsize\itshape}


\usepackage{natbib}
\bibliographystyle{plainnat}
\usepackage[strings]{underscore} % protect underscores in most circumstances



\newtheorem{hypothesis}{Hypothesis}
\usepackage{setspace}

\makeatletter
\@ifpackageloaded{hyperref}{}{%
\ifxetex
  \PassOptionsToPackage{hyphens}{url}\usepackage[setpagesize=false, % page size defined by xetex
              unicode=false, % unicode breaks when used with xetex
              xetex]{hyperref}
\else
  \PassOptionsToPackage{hyphens}{url}\usepackage[unicode=true]{hyperref}
\fi
}

\@ifpackageloaded{color}{
    \PassOptionsToPackage{usenames,dvipsnames}{color}
}{%
    \usepackage[usenames,dvipsnames]{color}
}
\makeatother
\hypersetup{breaklinks=true,
            bookmarks=true,
            pdfauthor={Haoyu Zhang (Department of Biostatistics, Johns Hopkins University, Baltimore, MD,
U.S.A.) and Ni Zhao (Department of Biostatistics, Johns Hopkins University, Baltimore, MD,
U.S.A.) and Thomas U. Ahearn (National Cancer Institute, Division of Cancer Epidemiology and Genetics,
Rockville, MD, U.S.A.) and William Wheeler (Information Management Services, Inc., Rockville, MD, U.S.A) and Montserrat García-Closas (National Cancer Institute, Division of Cancer Epidemiology and Genetics,
Rockville, MD, U.S.A.) and Nilanjan Chatterjee (Department of Biostatistics, Johns Hopkins University, Baltimore, MD,
USA)},
             pdfkeywords = {Two-stage polytomous model; Susceptibility variants; Cancer subtypes; EM
algorithm; Score tests; Etiologic heterogeneity.},  
            pdftitle={Two-stage polytmous logistic regression tutorial},
            colorlinks=true,
            citecolor=blue,
            urlcolor=blue,
            linkcolor=magenta,
            pdfborder={0 0 0}}
\urlstyle{same}  % don't use monospace font for urls

% set default figure placement to htbp
\makeatletter
\def\fps@figure{htbp}
\makeatother



% add tightlist ----------
\providecommand{\tightlist}{%
\setlength{\itemsep}{0pt}\setlength{\parskip}{0pt}}

\begin{document}
	
% \pagenumbering{arabic}% resets `page` counter to 1 
%
% \maketitle

{% \usefont{T1}{pnc}{m}{n}
\setlength{\parindent}{0pt}
\thispagestyle{plain}
{\fontsize{18}{20}\selectfont\raggedright 
\maketitle  % title \par  

}

{
   \vskip 13.5pt\relax \normalsize\fontsize{11}{12} 
\textbf{\authorfont Haoyu Zhang} \hskip 15pt \emph{\small Department of Biostatistics, Johns Hopkins University, Baltimore, MD,
U.S.A.}   \par \textbf{\authorfont Ni Zhao} \hskip 15pt \emph{\small Department of Biostatistics, Johns Hopkins University, Baltimore, MD,
U.S.A.}   \par \textbf{\authorfont Thomas U. Ahearn} \hskip 15pt \emph{\small National Cancer Institute, Division of Cancer Epidemiology and Genetics,
Rockville, MD, U.S.A.}   \par \textbf{\authorfont William Wheeler} \hskip 15pt \emph{\small Information Management Services, Inc., Rockville, MD, U.S.A}   \par \textbf{\authorfont Montserrat García-Closas} \hskip 15pt \emph{\small National Cancer Institute, Division of Cancer Epidemiology and Genetics,
Rockville, MD, U.S.A.}   \par \textbf{\authorfont Nilanjan Chatterjee} \hskip 15pt \emph{\small Department of Biostatistics, Johns Hopkins University, Baltimore, MD,
USA}   

}

}








\begin{abstract}

    \hbox{\vrule height .2pt width 39.14pc}

    \vskip 8.5pt % \small 

\noindent Cancers are routinely classified into subtypes according to various
features, including histo pathological characteristics and molecular
markers. Genomic investigations have reported heterogeneous association
between loci and cancer subtypes. However, it is not evident what is the
optimal modeling strategy for handling correlated tumor features,
missing data, and increased degrees-of-freedom in the underlying tests
of associations. In this tutorial, we proposed a two-stage polytomous
regression framework to handle cancer data with multivariate tumor
characteristics. In the first stage, a standard polytomous model is used
to specify for all subtypes defined by the cross-classification of
different markers. In the second stage, the subtype-specific
case-control odds ratios are specified using a more parsimonious model
based on the case-control odds ratio for a baseline subtype, and the
case-case parameters associated with tumor markers. Further, to reduce
the degrees-of-freedom, we allow to specify case-case parameters for
additional markers using a random-effect model. We use the EM algorithm
to account for missing data on tumor markers. The score-test
distribution theory is developed by borrowing analogous techniques from
group-based association tests. Through simulations across a range of
realistic scenarios, we show the proposed methods outperforms
alternative methods for identifying heterogenous associations between
risk loci and tumor subtypes.


\vskip 8.5pt \noindent \emph{Keywords}: Two-stage polytomous model; Susceptibility variants; Cancer subtypes; EM
algorithm; Score tests; Etiologic heterogeneity. \par

    \hbox{\vrule height .2pt width 39.14pc}



\end{abstract}


\vskip 6.5pt


\noindent  \section{Overview}\label{overview}

This vegnette provides an introduction to the `TOP' package. To load the
package, users need to install package from CRAN and TOP from github.
The package can be loaded with the following command:

\begin{Shaded}
\begin{Highlighting}[]
\CommentTok{# install.packages("devtools")}
\KeywordTok{library}\NormalTok{(devtools)  }
\CommentTok{#install_github("andrewhaoyu/TOP")}
\KeywordTok{library}\NormalTok{(TOP)}
\end{Highlighting}
\end{Shaded}

\section{Two-stage polytomous model}\label{two-stage-polytomous-model}

In this vegnette, we will decomstrate the methods with a breast cancer
example. There are around 5,112 cases and 4,888 controls in the dataset.
Four different tumor characteristics were included: ER (positive vs
negative), PR (positive vs negative), HER2 (positive vs negative), grade
(ordinal 0, 1, 2).

For simplicity, we will first demonstrate the two-stage model with three
binary tumor characteristics (ER, PR, and HER2). These three tumor
characteristics could define 8 different breast cancer subtypes
(8=2x2x2). We included two covariates, one is a SNP that we are
interested. The second one is the first principal component (PC1). Let
\(D_{i}\) denote the disease status, taking values in
\(\{0,1,2,\cdots,8\}\), of the \(i\)th (\(i \in 1, \cdots, 10,000\))
subject in the study. \(D_{i}=0\) represents a control, and \(D_{i}=m\)
represent a subject with disease of subtype \(m\). Let \(G_i\) be the
genotype for \(i\)th subject and \(X_i\) be the PC1 for the \(i\)th
subject. In the first-stage model, we use the standard ``saturated''
polytomous logistic regression model
\[ Pr(D_{i}=m|G_i,X_{i})=\frac{\exp(\beta_m G_i+\eta_m X_i)}{1+\sum_{m=1}^{8}\exp(\beta_m G_i+\eta_{m}X_{i})}\]
where \(\beta_{m}\) and \(\eta_{m}\) are the regression coefficients for
the SNP and PC1 for association with the \(m\)th subtype.

Because each cancer subtype \(m\) is defined through a unique
combination of the 3 characteristics, we can always alternatively index
the parameters \(\beta_m\) as \(\beta_{s_1s_2s_3}\), where
\(s_1,s_2,s_3\in\{0,1\}\) for the three binary tumor characteristics.
Originally \(\beta_1\) could be the coefficient of cancer subtype
ER-PR-HER2-. With the new index, \(\beta_1\) could be written as
\(\beta_{000}\), which means the ER, PR, HER2 are all negative. With
this new index, we can represent the log odds ratio as
\[\beta_{s_{1}s_{2}s_{3}}=\theta^{(0)}+\theta_{1}^{(1)}s_{1}+\theta_{2}^{(1)}s_{2}+\theta_{3}^{(1)}s_{3}+\theta_{12}^{(2)}s_{1}s_{2}+\theta_{13}^{(2)}s_{1}s_{3}+\theta_{23}^{(2)}s_{2}s_{3}
+\theta_{123}^{(3)}(s_{1}s_{2}s_{3}).\] Here \(\theta^{(0)}\) represents
the standard case-control log odds ratio for a reference disease subtype
compared to the control. \(\theta_{k}^{(1)}\) represents a case-case log
odds ratio associated with the levels of \(k\)th tumor characteristics
after adjusting for other tumor characteristics. We also refer
\(\theta_{k}^{(1)}\) as the main effect of the \(k\)th tumor
characteristic. And \(\theta_{k_1k_2}^{(2)}\) represents how the
case-case log odds ratio associated \(k_1\)th tumor characteristic is
modified by the levels of the \(k_2\)th tumor characteristic and vice
versa. We also refer \(\theta_{k_1k_2}^{(2)}\) as the pairwise
interaction between the \(k_1\)th and \(k_2\)th tumor characteristic.
And \(\theta_{123}^{(3)}\) represent the third order interactions of the
three tumor characteristics. Since both the first stage and second stage
have 8 parameters, this decomposition is equivalent with the first stage
polytomous logistic regression model. We called this saturated model.
The users could construct different two-stage model by assuming
different second stage parameters to be 0. For example, the baseline
only model: \[\beta_{s_{1}s_{2}s_{3}}=\theta^{(0)}.\] This model assumes
all of the subtypes have the same log odds ratio. So it is equivalent to
the standard logistic regression. We can also construct the additive
two-stage model by assuming all of the second stage interactions
parameters are 0, then the second stage decomposition becomes,
\[\beta_{s_{1}s_{2}s_{3}}=\theta^{(0)}+\theta_{1}^{(1)}s_{1}+\theta_{2}^{(1)}s_{2}+\theta_{3}^{(1)}s_{3}\]
Furthermore, we could construct the pairwise interaction two-stage model
by assuming all of the second stage higher order interactions parameters
are 0, then the second stage decomposition becomes,
\[\beta_{s_{1}s_{2}s_{3}}=\theta^{(0)}+\theta_{1}^{(1)}s_{1}+\theta_{2}^{(1)}s_{2}+\theta_{3}^{(1)}s_{3}+\theta_{12}^{(2)}s_{1}s_{2}+\theta_{13}^{(2)}s_{1}s_{3}+\theta_{23}^{(2)}s_{2}s_{3}\]

\begin{Shaded}
\begin{Highlighting}[]
\KeywordTok{library}\NormalTok{(TOP)}
\CommentTok{#load in the breast cancer example}
\KeywordTok{data}\NormalTok{(data, }\DataTypeTok{package=}\StringTok{"TOP"}\NormalTok{)}
\CommentTok{#this is a simulated breast cancer example}
\CommentTok{#there are around 5000 breast cancer cases and 5000 controls disease}
\NormalTok{data[}\DecValTok{1}\OperatorTok{:}\DecValTok{5}\NormalTok{,]}
\CommentTok{#four different tumor characteristics were included, }
\CommentTok{#ER (positive vs negative), }
\CommentTok{#PR (positive vs negative),}
\CommentTok{#HER2 (positive vs negative)}
\CommentTok{#the phenotype file}
\NormalTok{y <-}\StringTok{ }\NormalTok{data[,}\DecValTok{1}\OperatorTok{:}\DecValTok{4}\NormalTok{]}
\CommentTok{#one SNP}
\CommentTok{#one Principal components (PC1) are the covariates}
\NormalTok{SNP <-}\StringTok{ }\NormalTok{data[,}\DecValTok{6}\NormalTok{,drop=F]}
\NormalTok{PC1 <-}\StringTok{ }\NormalTok{data[,}\DecValTok{7}\NormalTok{,drop=F]}
\CommentTok{#fit the additive two-stage model}
\NormalTok{model.}\DecValTok{1}\NormalTok{ <-}\StringTok{ }\KeywordTok{TwoStageModel}\NormalTok{(}\DataTypeTok{y=}\NormalTok{y,}\DataTypeTok{additive=}\KeywordTok{cbind}\NormalTok{(SNP,PC1),}
                         \DataTypeTok{missingTumorIndicator =} \DecValTok{888}\NormalTok{)}
\end{Highlighting}
\end{Shaded}

\begin{Shaded}
\begin{Highlighting}[]
\CommentTok{#the model result is a list}
\CommentTok{#model.1[[4]] are the second stage odds ratio (95% CI) }
\CommentTok{#and p-value, the baseline effect is the case-control }
\CommentTok{#odds ratio of the reference subtype (ER-,PR-,HER2-,grade0). }
\CommentTok{#The main effect are the case-case odds ratio of the tumor characteristics.}
\NormalTok{(model.}\DecValTok{1}\NormalTok{[[}\DecValTok{4}\NormalTok{]])}
\end{Highlighting}
\end{Shaded}

\begin{verbatim}
##   Covariate SecondStageEffect OddsRatio OddsRatio(95%CI low)
## 1       SNP   baseline effect      0.97                 0.84
## 2       SNP    ER main effect      0.98                 0.87
## 3       SNP    PR main effect      1.06                 0.93
## 4       SNP  HER2 main effect      1.15                 0.99
## 5       PC1   baseline effect      1.13                 1.03
## 6       PC1    ER main effect      1.01                 0.93
## 7       PC1    PR main effect      0.95                 0.88
## 8       PC1  HER2 main effect      0.94                 0.86
##   OddsRatio(95%CI high)  Pvalue
## 1                  1.13 0.69200
## 2                  1.11 0.76700
## 3                  1.20 0.37000
## 4                  1.33 0.06380
## 5                  1.24 0.00755
## 6                  1.08 0.87100
## 7                  1.03 0.21500
## 8                  1.03 0.17700
\end{verbatim}

\begin{Shaded}
\begin{Highlighting}[]
\CommentTok{#model.1[[5]] are the global association test and }
\CommentTok{#global heterogeneity test result of the covariates.}
\NormalTok{model.}\DecValTok{1}\NormalTok{[[}\DecValTok{5}\NormalTok{]]}
\end{Highlighting}
\end{Shaded}

\begin{verbatim}
##   Covariate Global test for association Global test for heterogeneity
## 1       SNP                     0.10100                         0.248
## 2       PC1                     0.00921                         0.466
\end{verbatim}

\begin{Shaded}
\begin{Highlighting}[]
\CommentTok{#model.1[[7]] are the case-control odds ratios }
\CommentTok{#for all of the subtypes.}
\KeywordTok{head}\NormalTok{(model.}\DecValTok{1}\NormalTok{[[}\DecValTok{7}\NormalTok{]])}
\end{Highlighting}
\end{Shaded}

\begin{verbatim}
##   Covariate    Subtypes OddsRatio OddsRatio(95%CI low)
## 1       SNP ER0PR0HER20      0.97                 0.84
## 2       PC1 ER0PR0HER20      1.13                 1.03
## 3       SNP ER1PR0HER20      0.95                 0.81
## 4       PC1 ER1PR0HER20      1.14                 1.03
## 5       SNP ER0PR1HER20      1.03                 0.94
## 6       PC1 ER0PR1HER20      1.08                 1.02
##   OddsRatio(95%CI high)  Pvalue
## 1                  1.13 0.69200
## 2                  1.24 0.00755
## 3                  1.12 0.55900
## 4                  1.26 0.01010
## 5                  1.12 0.52700
## 6                  1.14 0.00623
\end{verbatim}

\begin{Shaded}
\begin{Highlighting}[]
\CommentTok{#instead of additive model, you can also try }
\CommentTok{#different combinations. For example, for the PC1,}
\CommentTok{#we use the additive model, but for SNP,}
\CommentTok{#we use the baseline only model.}
\NormalTok{model.}\DecValTok{2}\NormalTok{ <-}\StringTok{ }\KeywordTok{TwoStageModel}\NormalTok{(}\DataTypeTok{y=}\NormalTok{y,}\DataTypeTok{baselineonly =}\NormalTok{ SNP,}
                         \DataTypeTok{additive=}\NormalTok{PC1,}
                         \DataTypeTok{missingTumorIndicator =} \DecValTok{888}\NormalTok{)}
\end{Highlighting}
\end{Shaded}

The result is a list containing 9 elements. 1. the second stage
parameters 2. the covariance matrix for the second stage parameters. 3.
the second stage parameters organzied for each covariate. 4. The
case-control odds ratio and case-case odds ratios of tumor
characteristics. 5. Global association test and global heterogeneity
test result (Wald test based) 6. The first stage parameter organized for
each covariate 7. First stage odds ratio of all the subtypes. 8.
Likelihood 9. AIC

\section{Two-stage polytomous model self design second
stage}\label{two-stage-polytomous-model-self-design-second-stage}

Instead of using the hierarchical second stage decomposition we
discussed in last section, the two-stage model also allows the user to
self design the second stage matrix. For example, we could define five
intrinsic breast cancer subtypes based on the four tumor
characteristics: ER, PR, HER2, grade. The five intrinsic subtypes are:
1. (ER or PR)+, HER2-, grade 1 or 2; 2. (ER or PR)+, HER2+; 3. (ER or
PR)+, HER2-, grade 3; 4. (ER \& PR)-, HER2-; 5. ER-PR-HER2-. We are
interested in estimating the case-control log odds ratios of these
intrinsic subtypes.

\begin{Shaded}
\begin{Highlighting}[]
\KeywordTok{library}\NormalTok{(TOP)}
\CommentTok{#load in the breast cancer example}
\KeywordTok{data}\NormalTok{(data, }\DataTypeTok{package=}\StringTok{"TOP"}\NormalTok{)}
\CommentTok{#this is a simulated breast cancer example}
\CommentTok{#there are around 5000 breast cancer cases and 5000 controls disease}
\NormalTok{data[}\DecValTok{1}\OperatorTok{:}\DecValTok{5}\NormalTok{,]}
\CommentTok{#four different tumor characteristics were included, }
\CommentTok{#ER (positive vs negative), }
\CommentTok{#PR (positive vs negative),}
\CommentTok{#HER2 (positive vs negative)}
\CommentTok{#grade (oridinal 1,2,3)}
\CommentTok{#the phenotype file}
\NormalTok{y <-}\StringTok{ }\NormalTok{data[,}\DecValTok{1}\OperatorTok{:}\DecValTok{5}\NormalTok{]}
\CommentTok{#generate the combinations of all the subtypes}
\CommentTok{#by default, we remove all the subtypes with less than 10 cases}
\NormalTok{z.standard <-}\StringTok{ }\KeywordTok{GenerateZstandard}\NormalTok{(y)}
\NormalTok{M <-}\StringTok{ }\KeywordTok{nrow}\NormalTok{(z.standard) }\CommentTok{#M is the total number of first stage subtypes}

\CommentTok{#initial a z.design matrix with M rows, and 5 columns}
\CommentTok{#each row represent a first stage subtype}
\CommentTok{#each column represent an aggregated subtype}
\NormalTok{z.design <-}\StringTok{ }\KeywordTok{matrix}\NormalTok{(}\DecValTok{0}\NormalTok{,M,}\DecValTok{5}\NormalTok{)}
\CommentTok{#define names for the five intrinsic subtypes}
\KeywordTok{colnames}\NormalTok{(z.design) <-}\StringTok{ }\KeywordTok{c}\NormalTok{(}\StringTok{"HR+_HER2-_lowgrade"}\NormalTok{,}
                        \StringTok{"HR+_HER2+"}\NormalTok{,}
                        \StringTok{"HR+_HER2-_highgrade"}\NormalTok{,}
                        \StringTok{"HR-_HER2+"}\NormalTok{, }
                        \StringTok{"HR-_HER2-"}\NormalTok{)}
\CommentTok{#To construct a self design second stage matrix,}
\CommentTok{#we need to find the correpsonding first stage subtypes}
\CommentTok{#belonging to specific aggregated subtypes}
\CommentTok{#for first subtype HR+_HER2-_lowgrade}
\NormalTok{idx.}\DecValTok{1}\NormalTok{ <-}\StringTok{ }\KeywordTok{which}\NormalTok{((z.standard[,}\DecValTok{1}\NormalTok{]}\OperatorTok{==}\DecValTok{1}\OperatorTok{|}\NormalTok{z.standard[,}\DecValTok{2}\NormalTok{]}\OperatorTok{==}\DecValTok{1}\NormalTok{)}
               \OperatorTok{&}\NormalTok{z.standard[,}\DecValTok{3}\NormalTok{]}\OperatorTok{==}\DecValTok{0}
               \OperatorTok{&}\NormalTok{(z.standard[,}\DecValTok{4}\NormalTok{]}\OperatorTok{==}\DecValTok{1}\OperatorTok{|}\NormalTok{z.standard[,}\DecValTok{4}\NormalTok{]}\OperatorTok{==}\DecValTok{2}\NormalTok{))}
\NormalTok{z.design[idx.}\DecValTok{1}\NormalTok{,}\DecValTok{1}\NormalTok{] <-}\StringTok{ }\DecValTok{1}
\CommentTok{#for second subtype HR+_HER2+}
\NormalTok{idx.}\DecValTok{2}\NormalTok{ <-}\StringTok{ }\KeywordTok{which}\NormalTok{((z.standard[,}\DecValTok{1}\NormalTok{]}\OperatorTok{==}\DecValTok{1}\OperatorTok{|}\NormalTok{z.standard[,}\DecValTok{2}\NormalTok{]}\OperatorTok{==}\DecValTok{1}\NormalTok{)}
               \OperatorTok{&}\NormalTok{z.standard[,}\DecValTok{3}\NormalTok{]}\OperatorTok{==}\DecValTok{1}\NormalTok{)}
\NormalTok{z.design[idx.}\DecValTok{2}\NormalTok{,}\DecValTok{2}\NormalTok{] <-}\StringTok{ }\DecValTok{1}
\CommentTok{#for third subtype HR+_HER2-_highgrade}
\NormalTok{idx.}\DecValTok{3}\NormalTok{ <-}\StringTok{ }\KeywordTok{which}\NormalTok{((z.standard[,}\DecValTok{1}\NormalTok{]}\OperatorTok{==}\DecValTok{1}\OperatorTok{|}\NormalTok{z.standard[,}\DecValTok{2}\NormalTok{]}\OperatorTok{==}\DecValTok{1}\NormalTok{)}
               \OperatorTok{&}\NormalTok{z.standard[,}\DecValTok{3}\NormalTok{]}\OperatorTok{==}\DecValTok{0}
               \OperatorTok{&}\NormalTok{z.standard[,}\DecValTok{4}\NormalTok{]}\OperatorTok{==}\DecValTok{3}\NormalTok{)}
\NormalTok{z.design[idx.}\DecValTok{3}\NormalTok{,}\DecValTok{3}\NormalTok{] <-}\StringTok{ }\DecValTok{1}
\CommentTok{#for third subtype HR-_HER2+}
\NormalTok{idx.}\DecValTok{4}\NormalTok{ <-}\StringTok{ }\KeywordTok{which}\NormalTok{(z.standard[,}\DecValTok{1}\NormalTok{]}\OperatorTok{==}\DecValTok{0}\OperatorTok{&}\NormalTok{z.standard[,}\DecValTok{2}\NormalTok{]}\OperatorTok{==}\DecValTok{0}
               \OperatorTok{&}\NormalTok{z.standard[,}\DecValTok{3}\NormalTok{]}\OperatorTok{==}\DecValTok{1}\NormalTok{)}
\NormalTok{z.design[idx.}\DecValTok{4}\NormalTok{,}\DecValTok{4}\NormalTok{] <-}\StringTok{ }\DecValTok{1}
\CommentTok{#for third subtype HR-_HER2-}
\NormalTok{idx.}\DecValTok{5}\NormalTok{ <-}\StringTok{ }\KeywordTok{which}\NormalTok{(z.standard[,}\DecValTok{1}\NormalTok{]}\OperatorTok{==}\DecValTok{0}\OperatorTok{&}\NormalTok{z.standard[,}\DecValTok{2}\NormalTok{]}\OperatorTok{==}\DecValTok{0}
               \OperatorTok{&}\NormalTok{z.standard[,}\DecValTok{3}\NormalTok{]}\OperatorTok{==}\DecValTok{0}\NormalTok{)}
\NormalTok{z.design[idx.}\DecValTok{5}\NormalTok{,}\DecValTok{5}\NormalTok{] <-}\StringTok{ }\DecValTok{1}
\CommentTok{#one SNP and one Principal components (PC1) are the covariates}
\NormalTok{SNP <-}\StringTok{ }\NormalTok{data[,}\DecValTok{6}\NormalTok{,drop=F]}
\NormalTok{PC1 <-}\StringTok{ }\NormalTok{data[,}\DecValTok{7}\NormalTok{,drop=F]}

\NormalTok{model.}\DecValTok{3}\NormalTok{ <-}\StringTok{ }\KeywordTok{EMmvpolySelfDesign}\NormalTok{(y,}
          \DataTypeTok{x.self.design =}\NormalTok{ SNP,}
    \DataTypeTok{z.design =}\NormalTok{ z.design,}
    \DataTypeTok{additive=}\NormalTok{PC1,}
  \DataTypeTok{missingTumorIndicator =} \DecValTok{888}\NormalTok{)}
\end{Highlighting}
\end{Shaded}

\begin{Shaded}
\begin{Highlighting}[]
\CommentTok{#model.3[[4]] are the second stage odds ratio (95% CI) }
\CommentTok{#and p-value of the intrinsic subtypes}
\NormalTok{(model.}\DecValTok{3}\NormalTok{[[}\DecValTok{4}\NormalTok{]])}
\end{Highlighting}
\end{Shaded}

\begin{verbatim}
##   Covariate   SecondStageEffect OddsRatio OddsRatio(95%CI low)
## 1       SNP  HR+_HER2-_lowgrade      1.02                 0.93
## 2       SNP           HR+_HER2+      1.12                 1.02
## 3       SNP HR+_HER2-_highgrade      1.06                 0.93
## 4       SNP           HR-_HER2+      1.03                 0.73
## 5       SNP           HR-_HER2-      0.94                 0.66
##   OddsRatio(95%CI high) Pvalue
## 1                  1.12 0.7100
## 2                  1.22 0.0161
## 3                  1.22 0.3890
## 4                  1.44 0.8750
## 5                  1.36 0.7570
\end{verbatim}

\begin{Shaded}
\begin{Highlighting}[]
\CommentTok{#model.1[[5]] are the global association test and }
\CommentTok{#global heterogeneity test result of the covariates.}
\CommentTok{#Note global heterogeneity under self designed}
\CommentTok{#second stage matrix don't have interpretation}
\NormalTok{model.}\DecValTok{3}\NormalTok{[[}\DecValTok{5}\NormalTok{]]}
\end{Highlighting}
\end{Shaded}

\begin{verbatim}
##   Covariate Global test for association Global test for heterogeneity
## 1       SNP                       0.251                          0.16
\end{verbatim}

\begin{Shaded}
\begin{Highlighting}[]
\CommentTok{#model.1[[7]] are the case-control odds ratios }
\CommentTok{#for all of the subtypes.}
\KeywordTok{head}\NormalTok{(model.}\DecValTok{3}\NormalTok{[[}\DecValTok{7}\NormalTok{]])}
\end{Highlighting}
\end{Shaded}

\begin{verbatim}
##   Covariate          Subtypes OddsRatio OddsRatio(95%CI low)
## 1       SNP ER1PR0HER20grade1      1.02                 0.93
## 2       SNP ER0PR1HER20grade1      1.09                 0.97
## 3       SNP ER1PR1HER20grade1      1.02                 0.93
## 4       SNP ER0PR0HER21grade1      1.04                 0.97
## 5       SNP ER1PR0HER21grade1      1.02                 0.93
## 6       SNP ER0PR1HER21grade1      1.03                 0.94
##   OddsRatio(95%CI high) Pvalue
## 1                  1.12  0.710
## 2                  1.23  0.150
## 3                  1.12  0.710
## 4                  1.12  0.255
## 5                  1.12  0.710
## 6                  1.14  0.506
\end{verbatim}

\section{Fixed effect two-stage model score test
(FTOP)}\label{fixed-effect-two-stage-model-score-test-ftop}

To construct the score test for a fixed effect two stage model, we need
two steps. First, we need to fit the model under the null hypothesis
that the second stage parameters of SNP is 0. In other words, null of
the subytpes is associated with the SNP. Second we need should compute
the score and information matrix for each SNP. Based on the score and
information matrix, we could construct the score test statistics for
global association test

\begin{Shaded}
\begin{Highlighting}[]
\CommentTok{#fit the two-stage model under the null hypothesis}
\CommentTok{#that the second stage parameters of SNP is 0}
\CommentTok{#the model only has one covariate PC1}
\NormalTok{score.support.fixed <-}\StringTok{ }\KeywordTok{ScoreTestSupportMixedModel}\NormalTok{(}\DataTypeTok{y=}\NormalTok{y,}
                                                  \DataTypeTok{additive=}\NormalTok{PC1,}
                                                  \DataTypeTok{missingTumorIndicator=}\DecValTok{888}\NormalTok{)}
\end{Highlighting}
\end{Shaded}

\begin{verbatim}
## [1] "EM Algorithm Converged"
\end{verbatim}

\begin{Shaded}
\begin{Highlighting}[]
\CommentTok{#Generate the additive second stage design matrix}
\NormalTok{z.additive <-}\StringTok{ }\KeywordTok{cbind}\NormalTok{(}\DecValTok{1}\NormalTok{,z.standard)}
\CommentTok{#compute the score and information matrix for SNP}
\NormalTok{score.test.fixed <-}\StringTok{ }\KeywordTok{ScoreTestMixedModel}\NormalTok{(}\DataTypeTok{y=}\NormalTok{y,}
                                        \DataTypeTok{x=}\NormalTok{SNP,}
                                        \DataTypeTok{z.design =}\NormalTok{ z.additive,}
                                        \DataTypeTok{score.test.support=}\NormalTok{score.support.fixed,}
                                        \DataTypeTok{missingTumorIndicator=}\DecValTok{888}\NormalTok{)}
\CommentTok{#the first element is the score}
\CommentTok{#the second element is the information matrix}
\NormalTok{score.fixed <-}\StringTok{ }\NormalTok{score.test.fixed[[}\DecValTok{1}\NormalTok{]]}
\NormalTok{infor.fixed <-}\StringTok{ }\NormalTok{score.test.fixed[[}\DecValTok{2}\NormalTok{]]}
\CommentTok{#compute the global association test p value}
\NormalTok{p.value.ftop <-}\StringTok{ }\KeywordTok{DisplayFixedScoreTestResult}\NormalTok{(score.fixed,infor.fixed) }
\KeywordTok{print}\NormalTok{(p.value.ftop)}
\end{Highlighting}
\end{Shaded}

\begin{verbatim}
## [1] 0.0539
\end{verbatim}

\section{Mixed effect two-stage model score test
(MTOP)}\label{mixed-effect-two-stage-model-score-test-mtop}

To construct the score test for a mixed effect two stage model, we need
four steps. First, we need to fit the model under the null hypothesis
that the second stage parameters of SNP is 0. In other words, null of
the subytpes is associated with the SNP. Second, we need should compute
the score and information matrix of fixed effect for each SNP. Then, we
need to fit the model under the null that the variance of random effect
is 0. Finally, we need to compute the score and information matrix of
the random effect terms. With the score and information of fixed effect
and random effect, we could construct the global association test for
the mixed effect two-stage model.

\begin{Shaded}
\begin{Highlighting}[]
\CommentTok{#we are going to build a two-stage model with }
\CommentTok{#baseline parameter and ER case-case parameter as fixed}
\CommentTok{#We assume the PR, HER2, grade case-case parameter }
\CommentTok{#to be random}
\CommentTok{#fit the two-stage model under the null hypothesis}
\CommentTok{#that the second stage parameters of SNP is 0}
\CommentTok{#the model only has one covariate PC1}
\CommentTok{#Generate the z design matrix for fixed effect}
\NormalTok{z.design.fixed <-}\StringTok{ }\KeywordTok{cbind}\NormalTok{(}\DecValTok{1}\NormalTok{,z.standard[,}\DecValTok{1}\NormalTok{])}
\CommentTok{#compute the score and information matrix for fixed effect}
\NormalTok{score.test.fixed <-}\StringTok{ }\KeywordTok{ScoreTestMixedModel}\NormalTok{(}\DataTypeTok{y=}\NormalTok{y,}
                    \DataTypeTok{x=}\NormalTok{SNP,}
                    \DataTypeTok{z.design=}\NormalTok{z.design.fixed,}
                    \DataTypeTok{score.test.support=}\NormalTok{score.support.fixed,}
                    \DataTypeTok{missingTumorIndicator=}\DecValTok{888}\NormalTok{)}
\CommentTok{#the first element is the score}
\CommentTok{#the second element is the information matrix}
\NormalTok{score.fixed <-}\StringTok{ }\NormalTok{score.test.fixed[[}\DecValTok{1}\NormalTok{]]}
\NormalTok{infor.fixed <-}\StringTok{ }\NormalTok{score.test.fixed[[}\DecValTok{2}\NormalTok{]]}

\CommentTok{#fit the two-stage model under the null hypothesis}
\CommentTok{#that only the random effect is 0}
\CommentTok{#the model will have two covariates, }
\CommentTok{#PC1 and the fixed effect of SNP}
\NormalTok{score.support.random <-}\StringTok{ }\KeywordTok{ScoreTestSupportMixedModelSelfDesign}\NormalTok{(}\DataTypeTok{y=}\NormalTok{y,}
                        \DataTypeTok{x.self.design  =}\NormalTok{ SNP,}
                        \DataTypeTok{z.design =}\NormalTok{ z.design.fixed,}
                        \DataTypeTok{additive =}\NormalTok{ PC1,}
                        \DataTypeTok{missingTumorIndicator =} \DecValTok{888}\NormalTok{)}
\end{Highlighting}
\end{Shaded}

\begin{verbatim}
## [1] "EM Algorithm Converged"
\end{verbatim}

\begin{Shaded}
\begin{Highlighting}[]
\CommentTok{#Generate the z design matrix for random effect}
\CommentTok{#PR, HER2 and grade is random effect}
\NormalTok{z.design.random <-}\StringTok{ }\NormalTok{z.standard[,}\DecValTok{2}\OperatorTok{:}\DecValTok{4}\NormalTok{]}

\CommentTok{#compute the score and information matrix for random effect}
\NormalTok{score.test.random <-}\StringTok{ }\KeywordTok{ScoreTestMixedModel}\NormalTok{(}\DataTypeTok{y=}\NormalTok{y,}
                                         \DataTypeTok{x=}\NormalTok{SNP,}
                                         \DataTypeTok{z.design=}\NormalTok{z.design.random,}
                                         \DataTypeTok{score.test.support=}\NormalTok{score.support.random,}
                                         \DataTypeTok{missingTumorIndicator=}\DecValTok{888}\NormalTok{)}
\CommentTok{#the first element is the score}
\CommentTok{#the second element is the information matrix}
\NormalTok{score.random <-}\StringTok{ }\NormalTok{score.test.random[[}\DecValTok{1}\NormalTok{]]}
\NormalTok{infor.random <-}\StringTok{ }\NormalTok{score.test.random[[}\DecValTok{2}\NormalTok{]]}

\CommentTok{#after we get the the fixed effect score, infor }
\CommentTok{#and random effect score, infor, }
\CommentTok{#we can combine them through the following function. }
\CommentTok{#two p value will be generated.}
\CommentTok{#the first p value for global association test.}
\CommentTok{#the second p value is for the null hypothesis }
\CommentTok{#that random effect is 0}
\CommentTok{#Under this situation, the second p value is NOT }
\CommentTok{#the global heterogeneity test p value since ER is fixed}
\NormalTok{p.value.mtop <-}\StringTok{ }\KeywordTok{DisplayMixedScoreTestResult}\NormalTok{(}
\NormalTok{  score.fixed,}
\NormalTok{  infor.fixed,}
\NormalTok{  score.random,}
\NormalTok{  infor.random}
\NormalTok{)  }
\KeywordTok{print}\NormalTok{(p.value.mtop)}
\end{Highlighting}
\end{Shaded}

\begin{verbatim}
## [1] 0.09988282 0.14110544
\end{verbatim}

\section{References}\label{references}

\begin{enumerate}
\def\labelenumi{\arabic{enumi}.}
\tightlist
\item
  Zhang, H., Zhao, N., Ahearn, T.U, Wheeler W., García-Closas, M.,
  Chatterjee, N., A mixed-model approach for powerful testing of genetic
  associations with cancer risk incorporating tumor characteristics
  (Submitted)
\end{enumerate}




\newpage
\singlespacing 
\end{document}