\documentclass[11pt,]{article}
\usepackage[left=1in,top=1in,right=1in,bottom=1in]{geometry}
\newcommand*{\authorfont}{\fontfamily{phv}\selectfont}
\usepackage[]{mathpazo}


  \usepackage[T1]{fontenc}
  \usepackage[utf8]{inputenc}



\usepackage{abstract}
\renewcommand{\abstractname}{}    % clear the title
\renewcommand{\absnamepos}{empty} % originally center

\renewenvironment{abstract}
 {{%
    \setlength{\leftmargin}{0mm}
    \setlength{\rightmargin}{\leftmargin}%
  }%
  \relax}
 {\endlist}

\makeatletter
\def\@maketitle{%
  \newpage
%  \null
%  \vskip 2em%
%  \begin{center}%
  \let \footnote \thanks
    {\fontsize{18}{20}\selectfont\raggedright  \setlength{\parindent}{0pt} \@title \par}%
}
%\fi
\makeatother




\setcounter{secnumdepth}{0}

\usepackage{color}
\usepackage{fancyvrb}
\newcommand{\VerbBar}{|}
\newcommand{\VERB}{\Verb[commandchars=\\\{\}]}
\DefineVerbatimEnvironment{Highlighting}{Verbatim}{commandchars=\\\{\}}
% Add ',fontsize=\small' for more characters per line
\usepackage{framed}
\definecolor{shadecolor}{RGB}{248,248,248}
\newenvironment{Shaded}{\begin{snugshade}}{\end{snugshade}}
\newcommand{\KeywordTok}[1]{\textcolor[rgb]{0.13,0.29,0.53}{\textbf{#1}}}
\newcommand{\DataTypeTok}[1]{\textcolor[rgb]{0.13,0.29,0.53}{#1}}
\newcommand{\DecValTok}[1]{\textcolor[rgb]{0.00,0.00,0.81}{#1}}
\newcommand{\BaseNTok}[1]{\textcolor[rgb]{0.00,0.00,0.81}{#1}}
\newcommand{\FloatTok}[1]{\textcolor[rgb]{0.00,0.00,0.81}{#1}}
\newcommand{\ConstantTok}[1]{\textcolor[rgb]{0.00,0.00,0.00}{#1}}
\newcommand{\CharTok}[1]{\textcolor[rgb]{0.31,0.60,0.02}{#1}}
\newcommand{\SpecialCharTok}[1]{\textcolor[rgb]{0.00,0.00,0.00}{#1}}
\newcommand{\StringTok}[1]{\textcolor[rgb]{0.31,0.60,0.02}{#1}}
\newcommand{\VerbatimStringTok}[1]{\textcolor[rgb]{0.31,0.60,0.02}{#1}}
\newcommand{\SpecialStringTok}[1]{\textcolor[rgb]{0.31,0.60,0.02}{#1}}
\newcommand{\ImportTok}[1]{#1}
\newcommand{\CommentTok}[1]{\textcolor[rgb]{0.56,0.35,0.01}{\textit{#1}}}
\newcommand{\DocumentationTok}[1]{\textcolor[rgb]{0.56,0.35,0.01}{\textbf{\textit{#1}}}}
\newcommand{\AnnotationTok}[1]{\textcolor[rgb]{0.56,0.35,0.01}{\textbf{\textit{#1}}}}
\newcommand{\CommentVarTok}[1]{\textcolor[rgb]{0.56,0.35,0.01}{\textbf{\textit{#1}}}}
\newcommand{\OtherTok}[1]{\textcolor[rgb]{0.56,0.35,0.01}{#1}}
\newcommand{\FunctionTok}[1]{\textcolor[rgb]{0.00,0.00,0.00}{#1}}
\newcommand{\VariableTok}[1]{\textcolor[rgb]{0.00,0.00,0.00}{#1}}
\newcommand{\ControlFlowTok}[1]{\textcolor[rgb]{0.13,0.29,0.53}{\textbf{#1}}}
\newcommand{\OperatorTok}[1]{\textcolor[rgb]{0.81,0.36,0.00}{\textbf{#1}}}
\newcommand{\BuiltInTok}[1]{#1}
\newcommand{\ExtensionTok}[1]{#1}
\newcommand{\PreprocessorTok}[1]{\textcolor[rgb]{0.56,0.35,0.01}{\textit{#1}}}
\newcommand{\AttributeTok}[1]{\textcolor[rgb]{0.77,0.63,0.00}{#1}}
\newcommand{\RegionMarkerTok}[1]{#1}
\newcommand{\InformationTok}[1]{\textcolor[rgb]{0.56,0.35,0.01}{\textbf{\textit{#1}}}}
\newcommand{\WarningTok}[1]{\textcolor[rgb]{0.56,0.35,0.01}{\textbf{\textit{#1}}}}
\newcommand{\AlertTok}[1]{\textcolor[rgb]{0.94,0.16,0.16}{#1}}
\newcommand{\ErrorTok}[1]{\textcolor[rgb]{0.64,0.00,0.00}{\textbf{#1}}}
\newcommand{\NormalTok}[1]{#1}


\title{Two-stage polytmous logistic regression tutorial  }



\author{\Large Haoyu Zhang\vspace{0.05in} \newline\normalsize\emph{Department of Biostatistics, Johns Hopkins University, Baltimore, MD,
U.S.A.}   \and \Large Ni Zhao\vspace{0.05in} \newline\normalsize\emph{Department of Biostatistics, Johns Hopkins University, Baltimore, MD,
U.S.A.}   \and \Large Thomas U. Ahearn\vspace{0.05in} \newline\normalsize\emph{National Cancer Institute, Division of Cancer Epidemiology and Genetics,
Rockville, MD, U.S.A.}   \and \Large Montserrat García-Closas\vspace{0.05in} \newline\normalsize\emph{National Cancer Institute, Division of Cancer Epidemiology and Genetics,
Rockville, MD, U.S.A.}   \and \Large William Wheeler\vspace{0.05in} \newline\normalsize\emph{Information Management Services, Inc., Rockville, MD, U.S.A}   \and \Large Nilanjan Chatterjee\vspace{0.05in} \newline\normalsize\emph{Department of Biostatistics, Johns Hopkins University, Baltimore, MD,
USA}  }


\date{}

\usepackage{titlesec}

\titleformat*{\section}{\normalsize\bfseries}
\titleformat*{\subsection}{\normalsize\itshape}
\titleformat*{\subsubsection}{\normalsize\itshape}
\titleformat*{\paragraph}{\normalsize\itshape}
\titleformat*{\subparagraph}{\normalsize\itshape}


\usepackage{natbib}
\bibliographystyle{plainnat}
\usepackage[strings]{underscore} % protect underscores in most circumstances



\newtheorem{hypothesis}{Hypothesis}
\usepackage{setspace}

\makeatletter
\@ifpackageloaded{hyperref}{}{%
\ifxetex
  \PassOptionsToPackage{hyphens}{url}\usepackage[setpagesize=false, % page size defined by xetex
              unicode=false, % unicode breaks when used with xetex
              xetex]{hyperref}
\else
  \PassOptionsToPackage{hyphens}{url}\usepackage[unicode=true]{hyperref}
\fi
}

\@ifpackageloaded{color}{
    \PassOptionsToPackage{usenames,dvipsnames}{color}
}{%
    \usepackage[usenames,dvipsnames]{color}
}
\makeatother
\hypersetup{breaklinks=true,
            bookmarks=true,
            pdfauthor={Haoyu Zhang (Department of Biostatistics, Johns Hopkins University, Baltimore, MD,
U.S.A.) and Ni Zhao (Department of Biostatistics, Johns Hopkins University, Baltimore, MD,
U.S.A.) and Thomas U. Ahearn (National Cancer Institute, Division of Cancer Epidemiology and Genetics,
Rockville, MD, U.S.A.) and Montserrat García-Closas (National Cancer Institute, Division of Cancer Epidemiology and Genetics,
Rockville, MD, U.S.A.) and William Wheeler (Information Management Services, Inc., Rockville, MD, U.S.A) and Nilanjan Chatterjee (Department of Biostatistics, Johns Hopkins University, Baltimore, MD,
USA)},
             pdfkeywords = {Two-stage polytomous model; Susceptibility variants; Cancer subtypes; EM
algorithm; Score tests; Etiologic heterogeneity.},  
            pdftitle={Two-stage polytmous logistic regression tutorial},
            colorlinks=true,
            citecolor=blue,
            urlcolor=blue,
            linkcolor=magenta,
            pdfborder={0 0 0}}
\urlstyle{same}  % don't use monospace font for urls

% set default figure placement to htbp
\makeatletter
\def\fps@figure{htbp}
\makeatother



% add tightlist ----------
\providecommand{\tightlist}{%
\setlength{\itemsep}{0pt}\setlength{\parskip}{0pt}}

\begin{document}
	
% \pagenumbering{arabic}% resets `page` counter to 1 
%
% \maketitle

{% \usefont{T1}{pnc}{m}{n}
\setlength{\parindent}{0pt}
\thispagestyle{plain}
{\fontsize{18}{20}\selectfont\raggedright 
\maketitle  % title \par  

}

{
   \vskip 13.5pt\relax \normalsize\fontsize{11}{12} 
\textbf{\authorfont Haoyu Zhang} \hskip 15pt \emph{\small Department of Biostatistics, Johns Hopkins University, Baltimore, MD,
U.S.A.}   \par \textbf{\authorfont Ni Zhao} \hskip 15pt \emph{\small Department of Biostatistics, Johns Hopkins University, Baltimore, MD,
U.S.A.}   \par \textbf{\authorfont Thomas U. Ahearn} \hskip 15pt \emph{\small National Cancer Institute, Division of Cancer Epidemiology and Genetics,
Rockville, MD, U.S.A.}   \par \textbf{\authorfont Montserrat García-Closas} \hskip 15pt \emph{\small National Cancer Institute, Division of Cancer Epidemiology and Genetics,
Rockville, MD, U.S.A.}   \par \textbf{\authorfont William Wheeler} \hskip 15pt \emph{\small Information Management Services, Inc., Rockville, MD, U.S.A}   \par \textbf{\authorfont Nilanjan Chatterjee} \hskip 15pt \emph{\small Department of Biostatistics, Johns Hopkins University, Baltimore, MD,
USA}   

}

}








\begin{abstract}

    \hbox{\vrule height .2pt width 39.14pc}

    \vskip 8.5pt % \small 

\noindent Cancers are routinely classified into subtypes according to various
features, including histo pathological characteristics and molecular
markers. Genomic investigations have reported heterogeneous association
between loci and cancer subtypes. However, it is not evident what is the
optimal modeling strategy for handling correlated tumor features,
missing data, and increased degrees-of-freedom in the underlying tests
of associations. In this tutorial, we proposed a two-stage polytomous
regression framework to handle cancer data with multivariate tumor
characteristics. In the first stage, a standard polytomous model is used
to specify for all subtypes defined by the cross-classification of
different markers. In the second stage, the subtype-specific
case-control odds ratios are specified using a more parsimonious model
based on the case-control odds ratio for a baseline subtype, and the
case-case parameters associated with tumor markers. Further, to reduce
the degrees-of-freedom, we allow to specify case-case parameters for
additional markers using a random-effect model. We use the EM algorithm
to account for missing data on tumor markers. The score-test
distribution theory is developed by borrowing analogous techniques from
group-based association tests. Through simulations across a range of
realistic scenarios, we show the proposed methods outperforms
alternative methods for identifying heterogenous associations between
risk loci and tumor subtypes.


\vskip 8.5pt \noindent \emph{Keywords}: Two-stage polytomous model; Susceptibility variants; Cancer subtypes; EM
algorithm; Score tests; Etiologic heterogeneity. \par

    \hbox{\vrule height .2pt width 39.14pc}



\end{abstract}


\vskip 6.5pt


\noindent  \section{Overview}\label{overview}

This vegnette provides an introduction to the `TOP' package. To load the
package, users need to install package from CRAN and TOP from github.
The package can be loaded with the following command:

\begin{Shaded}
\begin{Highlighting}[]
\CommentTok{# install.packages("devtools")}
\KeywordTok{library}\NormalTok{(devtools)  }
\CommentTok{#install_github("andrewhaoyu/TOP")}
\KeywordTok{library}\NormalTok{(TOP)}
\end{Highlighting}
\end{Shaded}

\section{Two-stage polytomous model}\label{two-stage-polytomous-model}

In this vegnette, we will decomstrate the methods with a breast cancer
example. There are around 5,112 cases and 4,888 controls in the dataset.
Four different tumor characteristics were included: ER (positive vs
negative), PR (positive vs negative), HER2 (positive vs negative), grade
(ordinal 0, 1, 2).

For simplicity, we will first demonstrate the two-stage model with three
binary tumor characteristics (ER, PR, and HER2). These three tumor
characteristics could define 8 different breast cancer subtypes
(8=2x2x2). We included two covariates, one is a SNP that we are
interested. The second one is the first principal component (PC1). Let
\(D_{i}\) denote the disease status, taking values in
\(\{0,1,2,\cdots,8\}\), of the \(i\)th (\(i \in 1, \cdots, 10,000\))
subject in the study. \(D_{i}=0\) represents a control, and \(D_{i}=m\)
represent a subject with disease of subtype \(m\). Let \(G_i\) be the
genotype for \(i\)th subject and \(X_i\) be the PC1 for the \(i\)th
subject. In the first-stage model, we use the standard ``saturated''
polytomous logistic regression model
\[ Pr(D_{i}=m|G_i,X_{i})=\frac{\exp(\beta_m G_i+\eta_m X_i)}{1+\sum_{m=1}^{8}\exp(\beta_m G_i+\eta_{m}X_{i})}\]
where \(\beta_{m}\) and \(\eta_{m}\) is the regression coefficients for
the SNP and PC1 for association with the \(m\)th subtype.

Because each cancer subtype \(m\) is defined through a unique
combination of the 4 characteristics, we can always alternatively index
the parameters \(\beta_m\) as \(\beta_{s_1s_2s_3}\), where
\(s_1,s_2,s_3\in\{0,1\}\) for the three binary tumor characteristics.
Originally \(\beta_1\) could be the coefficient of cancer subtype
ER-PR-HER2-. With the new index, \(\beta_1\) could be written as
\(\beta_{000}\), which means the ER, PR, HER2 are all negative. With
this new index, we can represent the log odds ratio as
\[\beta_{s_{1}s_{2}s_{3}}=\theta^{(0)}+\theta_{1}^{(1)}s_{1}+\theta_{2}^{(1)}s_{2}+\theta_{3}^{(1)}s_{3}+\theta_{12}^{(2)}s_{1}s_{2}+\theta_{13}^{(2)}s_{1}s_{3}+\theta_{23}^{(2)}s_{2}s_{3}
+\theta_{123}^{(3)}(s_{1}s_{2}s_{3}).\] Here \(\theta^{(0)}\) represents
the standard case-control log odds ratio for a reference disease subtype
compared to the control and \(\theta_{k}^{(1)}\) represents a case-case
log odds ratio associated with the levels of \(k\)th tumor
characteristics after adjusting for other tumor characteristics, and
\(\theta_{k_1k_2}^{(2)}\) represent case-case log odds ratios associated
with pairwise interactions among the tumor characteristics and so on.
This decomposition is equivalent with the first stage model. Since both
the first stage and second stage have 8 parameters. We called this
saturated model. The users could construct different two-stage model by
assuming different second stage parameters to be 0. For example, the
baseline only model: \[\beta_{s_{1}s_{2}s_{3}}=\theta^{(0)}.\] This
model assumes all of the subtypes have the same log odds ratio. So it is
equivalent to the standard logistic regression. We can also construct
the additive two-stage model by assuming all of the second stage
interactions parameters are 0, then the second stage decomposition
becomes,
\[\beta_{s_{1}s_{2}s_{3}}=\theta^{(0)}+\theta_{1}^{(1)}s_{1}+\theta_{2}^{(1)}s_{2}+\theta_{3}^{(1)}s_{3}\]
Furthermore, we could construct the pairwise interaction two-stage model
by assuming all of the second stage higher order interactions parameters
are 0, then the second stage decomposition becomes,
\[\beta_{s_{1}s_{2}s_{3}}=\theta^{(0)}+\theta_{1}^{(1)}s_{1}+\theta_{2}^{(1)}s_{2}+\theta_{3}^{(1)}s_{3}+\theta_{12}^{(2)}s_{1}s_{2}+\theta_{13}^{(2)}s_{1}s_{3}+\theta_{23}^{(2)}s_{2}s_{3}\]

\begin{Shaded}
\begin{Highlighting}[]
\KeywordTok{library}\NormalTok{(TOP)}
\CommentTok{#load in the breast cancer example}
\KeywordTok{data}\NormalTok{(data, }\DataTypeTok{package=}\StringTok{"TOP"}\NormalTok{)}
\CommentTok{#this is a simulated breast cancer example}
\CommentTok{#there are around 5000 breast cancer cases and 5000 controls disease}
\NormalTok{data[}\DecValTok{1}\OperatorTok{:}\DecValTok{5}\NormalTok{,]}
\CommentTok{#four different tumor characteristics were included, }
\CommentTok{#ER (positive vs negative), }
\CommentTok{#PR (positive vs negative),}
\CommentTok{#HER2 (positive vs negative)}
\CommentTok{#the phenotype file}
\NormalTok{y <-}\StringTok{ }\NormalTok{data[,}\DecValTok{1}\OperatorTok{:}\DecValTok{4}\NormalTok{]}
\CommentTok{#one SNP}
\CommentTok{#one Principal components (PC1) are the covariates}
\NormalTok{SNP <-}\StringTok{ }\NormalTok{data[,}\DecValTok{6}\NormalTok{,drop=F]}
\NormalTok{PC1 <-}\StringTok{ }\NormalTok{data[,}\DecValTok{7}\NormalTok{,drop=F]}
\CommentTok{#fit the additive two-stage model}
\NormalTok{model.}\DecValTok{1}\NormalTok{ <-}\StringTok{ }\KeywordTok{TwoStageModel}\NormalTok{(}\DataTypeTok{y=}\NormalTok{y,}\DataTypeTok{additive=}\KeywordTok{cbind}\NormalTok{(SNP,PC1),}
                         \DataTypeTok{missingTumorIndicator =} \DecValTok{888}\NormalTok{)}
\end{Highlighting}
\end{Shaded}

\begin{Shaded}
\begin{Highlighting}[]
\CommentTok{#the model result is a list}
\CommentTok{#model.1[[4]] are the second stage odds ratio (95% CI) }
\CommentTok{#and p-value, the baseline effect is the case-control }
\CommentTok{#odds ratio of the reference subtype (ER-,PR-,HER2-,grade0). }
\CommentTok{#The main effect are the case-case odds ratio of the tumor characteristics.}
\NormalTok{(model.}\DecValTok{1}\NormalTok{[[}\DecValTok{4}\NormalTok{]])}
\end{Highlighting}
\end{Shaded}

\begin{verbatim}
##   Covariate SecondStageEffect OddsRatio OddsRatio(95%CI low)
## 1       SNP   baseline effect      0.97                 0.84
## 2       SNP    ER main effect      0.98                 0.87
## 3       SNP    PR main effect      1.06                 0.93
## 4       SNP  HER2 main effect      1.15                 0.99
## 5       PC1   baseline effect      1.13                 1.03
## 6       PC1    ER main effect      1.01                 0.93
## 7       PC1    PR main effect      0.95                 0.88
## 8       PC1  HER2 main effect      0.94                 0.86
##   OddsRatio(95%CI high)  Pvalue
## 1                  1.13 0.69200
## 2                  1.11 0.76700
## 3                  1.20 0.37000
## 4                  1.33 0.06380
## 5                  1.24 0.00755
## 6                  1.08 0.87100
## 7                  1.03 0.21500
## 8                  1.03 0.17700
\end{verbatim}

\begin{Shaded}
\begin{Highlighting}[]
\CommentTok{#model.1[[5]] are the global association test and }
\CommentTok{#global heterogeneity test result of the covariates.}
\NormalTok{model.}\DecValTok{1}\NormalTok{[[}\DecValTok{5}\NormalTok{]]}
\end{Highlighting}
\end{Shaded}

\begin{verbatim}
##   Covariate Global test for association Global test for heterogeneity
## 1       SNP                     0.10100                         0.248
## 2       PC1                     0.00921                         0.466
\end{verbatim}

\begin{Shaded}
\begin{Highlighting}[]
\CommentTok{#model.1[[7]] are the case-control odds ratios }
\CommentTok{#for all of the subtypes.}
\KeywordTok{head}\NormalTok{(model.}\DecValTok{1}\NormalTok{[[}\DecValTok{7}\NormalTok{]])}
\end{Highlighting}
\end{Shaded}

\begin{verbatim}
##   Covariate    Subtypes OddsRatio OddsRatio(95%CI low)
## 1       SNP ER0PR0HER20      0.97                 0.84
## 2       PC1 ER0PR0HER20      1.13                 1.03
## 3       SNP ER1PR0HER20      0.95                 0.81
## 4       PC1 ER1PR0HER20      1.14                 1.03
## 5       SNP ER0PR1HER20      1.03                 0.94
## 6       PC1 ER0PR1HER20      1.08                 1.02
##   OddsRatio(95%CI high)  Pvalue
## 1                  1.13 0.69200
## 2                  1.24 0.00755
## 3                  1.12 0.55900
## 4                  1.26 0.01010
## 5                  1.12 0.52700
## 6                  1.14 0.00623
\end{verbatim}

\begin{Shaded}
\begin{Highlighting}[]
\CommentTok{#instead of additive model, you can also try }
\CommentTok{#different combinations. For example, for the PC1,}
\CommentTok{#we use the additive model, but for SNP,}
\CommentTok{#we use the baseline only model.}
\NormalTok{model.}\DecValTok{2}\NormalTok{ <-}\StringTok{ }\KeywordTok{TwoStageModel}\NormalTok{(}\DataTypeTok{y=}\NormalTok{y,}\DataTypeTok{baselineonly =}\NormalTok{ SNP,}
                         \DataTypeTok{additive=}\NormalTok{PC1,}
                         \DataTypeTok{missingTumorIndicator =} \DecValTok{888}\NormalTok{)}
\end{Highlighting}
\end{Shaded}

The results of the function contain three elements: 1. p.dir is the
p-value of likelihood ratio test based on emprical distrition. 2. p.aud
is the p-value by approximating the null distribution as a mixture of a
point mass at zero with probability b and weighted chi square
distribution with d degrees of freedom with probality of 1-b. 3. LR is
the likelihood ratio test statistics.

\section{References}\label{references}

\begin{enumerate}
\def\labelenumi{\arabic{enumi}.}
\tightlist
\item
  N. Zhao, H. Zhang, J. Clark, A. Maity, M. Wu. Composite Kernel Machine
  Regression based on Likelihood Ratio Test with Application for
  Combined Genetic and Gene-environment Interaction Effect (Submitted)
\end{enumerate}




\newpage
\singlespacing 
\end{document}